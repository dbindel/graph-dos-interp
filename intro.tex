\section{Introduction}
\label{sec:intro}

Researchers from many disciplines study spectra to understand the
structure and composition of mathematical objects and physical systems.
For large systems, {\em complete} spectral information costs too much,
whether it is gathered computationally or measured experimentally;
hence, most spectral analysis methods use {\em partial} information
about the eigenvalues and eigenvectors of some matrix or operator.
Broadly speaking, these methods use either
\begin{itemize}
\item
  a few (extreme) eigenvalues and associated eigenvectors;
\item
  invariants that are simple functions of all the eigenvalues; or
\item densities of eigenvalues ({\em aka} the density of states),
  possibly weighted by how important they are to a particular node
  (i.e.~a local density of states).
\end{itemize}
All three approaches are used in spectral geometry, spectral graph
theory, and in applications to physics and engineering systems. But in
the study of complex networks, information about spectral
densities is much less often used.

These notes describe our efforts to apply to network science the local
and global ``density of states'' spectral analysis techniques common in
other areas.
